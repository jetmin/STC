\documentclass[titlepage,a4paper,12pt]{article}

\usepackage{amsmath,amsfonts,amssymb,amsthm,mathrsfs,enumitem}
\usepackage[french]{babel}
\usepackage{comment}
\usepackage{tikz}
\usepackage[utf8]{inputenc}
\usepackage{centernot}
\usepackage[parfill]{parskip}
\newcounter{d}
\newcounter{t}
\newcounter{p}
\newcounter{c}
\newcounter{a}
\newcounter{l}
\frenchbsetup{StandardLists = true}

\makeatletter
\def\thm@space@setup{%
  \thm@preskip=5pt \thm@postskip=0pt
}
\makeatother

\newtheorem{defi}[d]{Définition}
\newtheorem{prop}[p]{Proposition}
\newtheorem{algo}[a]{Algorithme}
\newtheorem{thm}[t]{Théorème}
\newtheorem{lem}[l]{Lemme}

\begin{document}
\section{chemin espace-temps}
Nous considérons les sommets et les arêtes qui inclus dans une boite $\Lambda(l)$ de taille $2l\times h$. On étudie les chemin espace-tempss fermés dans la boîte formés par la percolation dynamique de paramètre $p$.
\begin{defi}
Une arête-temps est un couple $(e,t)$ où $e$ est une arête de $\mathbb{E}$ et $t$ un nombre réel. 
\end{defi}

Nous définissons une relation d'équivalence de connexion sur l'espace $\mathbb{E}\times \mathbb{R}$ de la manière suivante: nous disons que les arêtes-temps $(e,t)$ et $(f,s)$ sont connectés si $e=f$ ou $(s=t$ et $e\sim f)$. Nous notons $(e,t)\sim(f,s)$ si l'une des conditions est vérifiée. Un chemin espace-temps est une suite d'arête-temps $(e_i,t_i)_{i\geqslant 0}$ telle que pour tout $i\geqslant 0$, $(e_i,t_i)\sim(e_{i+1},t_{i+1})$. Désormais, nous considérons les chemins espace-temps simple, i.e. si $e_k = e_{k+1}$ alors, $e_{k+2} \neq e_k$.

Nous considérons le processus de percolation dynamique à temps discret. L'espace des trajectoires 
Nous appelons un chemin espace-temps d'occurrence disjointe de longueur $n$ avec $m$ changement de temps s'il existe $m$ indices $1\leqslant k(1)< k(2) < \dots < k(m) \leqslant n$ telles que:
\begin{itemize}[label = $\bullet$, leftmargin = *]
\item Les changements de temps arrivent aux instants $t_{k(1)},\dots, t_{k(m)}$, i.e.
$$\forall i\in \{1,\dots, m-1\} \qquad e_{k(i)} = e_{k(i)+1} \quad t_{k(i)+1} =\dots = t_{k(i+1)}.
$$
\item les arêtes visitées à un temps donné sont 2 à 2 distinctes, i.e.
$$ \forall i,j \in \{1,\dots, n\} \qquad i\neq j \Rightarrow e_i\neq e_j.
$$
\item les fermetures d'arêtes arrivent disjointement, i.e. pour tout $i,j\in \{1,\dots, j\}$, $i<j$ tels que $e_i = e_j$, l'une des 3 conditions suivantes est vérifiée:
\begin{itemize}
\item $j=i+1$ et $i\in \{k(1),\dots, k(m)\}$;
\item $t_i< t_j$ et il existe un instant $s\in ]t_i,t_j[$ tel que $e_j$ est ouverte à $s$;
\item $t_j< t_i$ et il existe un instant $s\in ]t_j,t_i[$ tel que $e_j$ est ouverte à $s$;
\end{itemize}
\end{itemize}

\begin{prop}
Soit $(e_1,t_1),\dots,(e_N,t_N)$ un chemin espace-temps qui relie $x$ à $y$, il existe une fonction $\phi: \{1,\dots, n\}\rightarrow \{1,\dots, N\}$ strictement croissante telle que $(e_{\phi(1)},t_{\phi(1)}),\dots,(e_{\phi(n)},t_{\phi(n)})$ est un chemin espace-temps d'occurrence disjointe qui relie $x$ à $y$.
\end{prop}
\begin{proof}
Nous allons montrer cette proposition par récurrence sur la longueur $N$. Supposons que la proposition est vrai pour tout chemin de longueur inférieur à $N$. Nous considérons maintenant un chemin $$(e_1,t_1),\dots, (e_{N+1},t_{N+1})$$ de longueur $N+1$ qui relie $x$ $y$. S'il existe une indice $i\leqslant N$ telle que $(e_i,t_i) = (e_{N+1},t_{N+1})$ alors le chemin $$(e_1,t_1),\dots,(e_i,t_i)
$$ est un chemin de longueur $i\leqslant N$ qui relie $x$ à $y$. Par l'hypothèse de récurrence, nous avons un chemin extrait d'occurrence disjointe qui relie $x$ à $y$. 
S'il existe une indice $1\leqslant i\leqslant N$ telles que $e_i = e_{N+1}$, et $e_{N+1}$ reste fermée entre $t_i$ et $t_{N+1}$, nous considérons le chemin $$(e_1,t_1), \dots, (e_{i},t_{i}), (e_{N+1},t_{N+1})$$ qui est de longueur inférieur à $N$. Nous appliquons l'hypothèse de récurrence à ce chemin et nous obtenons le chemin extrait. Si aucun des cas précédents se présente, nous considérons le chemin $(e_1,t_1),\dots,(e_N,t_N)$ de longueur $N$ et qui relie $x$ à $z$. Par l'hypothèse de récurrence, il existe une fonction strictement croissante $\phi:\{1,\dots,n\}\rightarrow \{1,\dots, N\}$ telle que le chemin extrait $\gamma(\phi) = (e_{\phi(1)},t_{\phi(1)}),\dots,(e_{\phi(n)},t_{\phi(n)})$ est un chemin d'occurrence disjointe qui relie $x$ à $z$. 

Si ce chemin n'emprunte pas l'arête $e_{N+1}$, alors nous posons $\phi(n+1) = N+1$ et nous obtenons le chemin extrait souhaité.

Considérons le cas où $\gamma(\phi)$ emprunte l'arête $e_{N+1}$. Supposons tout d'abord que $\gamma(\phi)$ passe par $e_{N+1}$ avant et après $t_{N+1}$. Nous considérons $t_-$ (resp. $t_+$) le dernier (resp. premier) instant strictement avant (resp. après) $t_{N+1}$ où $\gamma(\phi)$ visite $e_{N+1}$ et soit $j_-$ (resp. $j_+$) l'unique indice telle que $t_{\phi(j_-)}= t_-$ (resp. $t_{\phi(j_+)}= t_+$) et $e_{\phi(j_-)}= e_{N+1}$ (resp. $e_{\phi(j_+)}= e_{N+1}$). Plus formellement, les indices $j_-,j_+$ sont définis par les conditions suivantes:
\begin{align*}
{j_-}< t_{N+1}\quad e_{j_-} = e_{N+1}\quad t_{j_-} = \max \big\{\,t_j:1\leqslant j \leqslant N, e_j = e_{N+1}, t_j < t_{N+1}\,\big\}\\
{j_+}> t_{N+1}\quad e_{j_+} = e_{N+1}\quad t_{j_+} = \min \big\{\,t_j:1\leqslant j \leqslant N, e_j = e_{N+1}, t_j > t_{N+1}\,\big\}
\end{align*}
Comme $\gamma(\phi)$ est d'occurrence disjointe qui ne contient pas l'arête temps $(e_{N+1},t_{N+1})$, et qu'aucun de ses changements de temps ne contient pas cette arête-temps, nécessairement, l'arête $e_{N+1}$ doit s'ouvrir dans l'intervalle $]t_{j^-},t_{j^+}[$.

Si l'arête $e_{N+1}$ s'ouvre sur $]t_{j_-},t_{N+1}[$ et sur $]t_{N+1},t_{j_+}[$, nous ajoutons $(e_{N+1},t_{N+1})$ à la fin de $\gamma(\phi)$ et nous obtenons le chemin extrait d'occurrence disjointe qui relie $x$ à $y$.

Si l'arête reste fermée sur $]t_{j_-},t_{N+1}[$, nécessairement, elle s'ouvre sur $]t_{N+1},t_{j_+}[$. Nous considérons le chemin espace-temps 
$$(e_{\phi(1)},t_{\phi(1)}), \dots, (e_{\phi(j_-)},t_{\phi(j_-)}),(e_{N+1},t_{N+1}).
$$
Ce chemin est d'occurrence disjointe et il relie $x$ à $y$.
Le cas où $e_{N+1}$ reste fermée sur $]t_{N+1},t_{j_+}[$ se traite de manière similaire en remplaçant $j_-$ par $j_+$.

Maintenant, nous supposons que $\gamma(\phi)$ visite $e_{N+1}$ uniquement avant $t_{N+1}$, nous définissons $j_-$ de la même façon que dans le cas précédent. Si $e_{N+1}$ reste fermée entre $]t_{j_-}, t_{N+1}[$, nous considérons le chemin extrait $$(e_{\phi(1)},t_{\phi(1)}),\dots, (e_{\phi(j_-)},t_{\phi(j_-)}),(e_{N+1},t_{N+1}).$$ Il est d'occurrence disjointe et il relie $x$ à $y$. Si $e_{N+1}$ s'ouvre entre $t_{j_-}$ et $t_{N+1}$, alors le chemin $(e_{\phi(1)},t_{\phi(1)}),\dots, (e_{\phi(n)},t_{\phi(n)}),(e_{N+1},t_{N+1})$ vérifie les conditions voulues.

Enfin, si $\gamma(\phi)$ visite $e_{N+1}$ uniquement après $t_{N+1}$, nous définissons seulement $j_+$ et nous obtenons le chemin extrait voulu de la même manière que le cas précédent.
\end{proof}

\begin{defi}
Un chemin espace-temps est dit impatient si toute arête de changement de temps $e_k$ est suivi par une arête $e_{k+2}$ qui change son état à l'instant $t_{k+2}$.
\end{defi}

Nous montrons tout chemin espace-temps admet une modification temporelle qui est impatiente. Plus formellement, nous introduisons l'algorithme de modification récursive suivante:
\begin{algo} soit $(e_1,t_1),\dots,(e_n,t_n)$ un chemin espace-temps, nous allons modifier la première arête $e_1$ du chemin, selon les cas suivants:
\begin{itemize}[label = $\bullet$, leftmargin = *]
\item si $e_2 \neq e_1$, alors nécessairement $t_1 = t_2$, et nous ne modifions pas $(e_1,t_1)$ et nous recommençons l'algorithme avec le chemin $(e_2,t_2),\dots,(e_n,t_n)$;
\item  si $t_1< t_2$, soit $\tau_3$ le dernier instant avant $t_2$ où $e_{3}$ se ferme. Si $t_1 \geqslant \tau_3$, nous remplaçons $(e_1,t_1),(e_2,t_2)$ par $(e_1,t_1),(e_3,t_1)$ et nous recommençons avec le chemin $(e_3,t_1),(e_3,t_3),\dots, (e_n,t_n)$; sinon $t_1 < \tau_3$, nous remplaçons $(e_1,t_1),(e_2,t_2)$ par $(e_1,t_1),(e_2,\tau_3),(e_3,\tau_3)$ et nous recommençons l'algorithme avec le chemin $(e_3,\tau_3),(e_3,t_3),\dots,(e_n,t_n)$;
\item si $t_1 > t_2$, soit $\tau_3$ le premier instant après $t_2$ où $e_{3}$ s'ouvre. Si $t_1 \leqslant \tau_3$, nous remplaçons $(e_1,t_1),(e_2,t_2)$ par $(e_1,t_1),(e_3,t_1)$ et nous recommençons avec le chemin $(e_3,t_1),(e_3,t_3),\dots, (e_n,t_n)$; sinon $t_1 > \tau_3$, nous remplaçons $(e_1,t_1),(e_2,t_2)$ par $(e_1,t_1),(e_2,\tau_3),(e_3,\tau_3)$ et nous recommençons l'algorithme avec le chemin $(e_3,\tau_3),(e_3,t_3),\dots,(e_n,t_n)$.
\end{itemize}
\end{algo}
Nous remarquons que la longueur de chemin non modifié diminue après chaque itération, donc l'algorithme se termine. Par la définition d'un chemin impatient, nous avons directement la propriété suivante:
\begin{prop}
Soit $(e_1,t_1),\dots,(e_n,t_n)$ un chemin espace-temps, sa modification obtenue selon l'algorithme précédent est impatient.
\end{prop}
Nous montrons maintenant qu'un chemin d'occurrence disjointe est toujours d'occurrence disjointe après la modification selon l'algorithme.
\begin{prop} Soit $\gamma = (e_1,t_1),\dots,(e_n,t_n)$ un chemin espace-temps d'occurrence disjointe, nous modifions ce chemin selon algorithme 1. Le chemin obtenu est d'occurrence disjointe et impatient.
\end{prop}
\begin{proof}
Nous vérifions la condition de l'occurrence disjointe dans chaque étape de l'algorithme. Soit $(e_i,t_i),(e_{i+1},t_{i+1})$ le changement de temps qui est modifié après une itération, et supposons que le chemin visite $e_i$ ou $e_{i+2}$ plus qu'une fois. Nous pouvons aussi supposer que $t_i< t_{i+1}$ car le cas $t_i> t_{i+1}$ se traite de la même manière. Nous examinons les deux résultats de la modification: si nous obtenons $(e_i,t_i),(e_{i+2},t_i)$, il faut simplement vérifier qu'il existe un instant entre chaque visite de $e_{i+2}$ et $t_i$ tel que $e_{i+2}$ est ouverte à cette instant. Or $(e_{i+2},t_{i+2})$ est dans $\gamma$ qui est un chemin d'occurrence disjointe, $e_{i+2}$ admet une ouverture entre les autres instants de visites et $t_{i+2}$. Vu que l'arête $e_{i+2}$ est fermée entre $t_i$ et $t_{i+2}$, de même pour $t_i$. Si nous avons $(e_i,t_i),(e_{i+1},\tau_{i+2}),(e_{i+2},\tau_{i+2})$ après la modification, nous devons vérifier la condition pour $e_i$ et $e_{i+2}$. Nous rappelons que $e_{i+1}= e_i$ et $\tau_{i+2}$ le dernier instant avant $t_{i+1}$ où $e_{i+2}$ se ferme. Or $e_i$ est fermée entre $t_i$ et $\tau_{i+2}$ et $e_{i+2}$ est fermée entre $\tau_{i+2}$ et $t_{i+2}$, nous avons le résultat voulu comme dans le premier cas. 
\end{proof}

\section{Décroissance exponentielle}
Nous démontrons maintenant la décroissance exponentielle de la probabilité d'avoir un chemin espace-temps qui relie deux points de distance $l$. Nous notons $\smash{x\overset{s,t}{\longleftrightarrow} y}$ l'événement qu'il existe un chemin espace temps fermé qui relie $x$ et $y$ dans un intervalle de temps $[s,t]$.
Avant d'énoncer l'estimation, nous montrons un lemme combinatoire:
\begin{lem} Soit $S(n,m)$ l'ensemble de $m$-uplet d'entiers entre $1$ et $n$ croissant qui n'admet pas d'entiers consécutifs $$S(n,m)=\{\,(u_1,\dots,u_m)\in \{1,\dots,n\}^m:\quad \forall 1\leqslant i\leqslant m-1,u_{i+1}>u_i+1 \, \},$$ alors $|S(n,m)| = \binom{n-m+1}{m}$.
\end{lem}
\begin{proof}
Nous considérons l'application suivante:
$$ \Phi : (u_1,\dots,u_m) \rightarrow (u_1,\dots,u_i-i+1,\dots,u_m-m+1).
$$
Il est facile de vérifier que $\Phi$ est une bijection entre $S(n,m)$ et l'ensemble de $m$-uplet strictement croissant entre $1$ et $n-m+1$. Ce dernier est de cardinal $\binom{n-m+1}{m}$.
\end{proof}
\begin{thm} Soit $\gamma$ un chemin qui réalise $\smash{x\overset{s,t}{\longleftrightarrow} y}$, et supposons que $\gamma$ est d'occurrence disjointe et impatient. Nous avons:
$$P\left(\begin{array}{c}
x\overset{s,t}{\longleftrightarrow} y \text{ par }\gamma\\
|\gamma| = n
\end{array}\right) \leqslant C\exp\big(\frac{2n(t-s)}{|\Lambda|}+\frac{n}{2}\ln(3-3p)\big)$$ avec $C$ une constante uniforme.
\end{thm}
\begin{proof}
Notons $(e_1,t_1),\dots,(e_n,t_n)$ les arêtes temps de $\gamma$, $k(1),\dots,k(m)$ les indices de changement de temps a lieu nous notons par convention $k(0) = 1$ et $k(m+1)=n$. Notons aussi pour tout $0\leqslant i \leqslant m$, $\mathcal{E}_i$ l'événement il existe n chemin fermé qui relie une extrémité de $e_{k(i)}$ à une extrémité de $e_{k(i+1)}$ à l'instant $t_{k(i+1)}$, $e_{k(i)}$ reste fermée entre $t_{k(i)}$ et $t_{k(i+1)}$ et $e_{k(i)+2}$ se ferme à l'instant $t_{k(i+1)}$. Nous factorisons la probabilité à l'aide de l'inégalité de BK:
\begin{align*}
&P\left(\begin{array}{c}
x\overset{s,t}{\longleftrightarrow} y \text{ par }\gamma\\
|\gamma| = n
\end{array}\right)= \sum_{m=0}^{\lfloor\frac{n}{2}\rfloor}\sum_{1=k(0)<\dots<k(m+1)=n}\sum_{t_{k(1)},\dots,t_{k(m)}}P(\mathcal{E}_0\circ\dots\circ \mathcal{E}_m)\\
&\leqslant \sum_{m=0}^{\lfloor\frac{n}{2}\rfloor}\sum_{1=k(0)<\dots<k(m+1)=n}\sum_{t_{k(1)},\dots,t_{k(m)}} \prod_{i=0}^mP(\mathcal{E}_i).
\end{align*}
Nous calculons maintenant chaque terme $P(\mathcal{E}_i)$ en notant $e_{k(i)}= \langle x_i,y_i\rangle$.
\begin{align*}
P(\mathcal{E}_i) &= P\left(\begin{array}{c}
y_i\longleftrightarrow x_{i+1}\text{ à l'instant }t_{k(i+1)}\\
e_{k(i)} \text{ reste fermée entre } t_{k(i)} \text{ et } t_{k(i+1)}\\
e_{k(i)+2} \text{ change d'état à }t_{k(i+1)}
\end{array}\right)\\
& \leqslant (3-3p)^{k(i+1)-k(i)-1)}(1-\frac{p}{|\Lambda|})^{|t_{k(i+1)}-t_{k(i)}|}\frac{1}{|\Lambda|}
\end{align*}
Nous appliquons la majoration dans la première probabilité et nous avons:
\begin{multline*}
P\left(\begin{array}{c}
x\overset{s,t}{\longleftrightarrow} y \text{ par }\gamma\\
|\gamma| = n
\end{array}\right) \leqslant \sum_{m=0}^{\lfloor\frac{n}{2}\rfloor}\sum_{1=k(0)<\dots<k(m+1)=n} \sum_{t_{k(1)},\dots,t_{k(m)}}\\(3-3p)^{n-m}(1-\frac{p}{|\Lambda|})^{\sum_i^{m-1}|t_{k(i+1)}-t_{k(i)}|}\frac{1}{|\Lambda|^m}
\end{multline*}
Calculons d'abord le terme qui contient les instants $t_{k(1)},\dots,t_{k(m)}$, en posant $\Delta_i = |t_{k(i+1)}-t_{k(i)}|$:
$$\sum_{t_{k(1)},\dots,t_{k(m)}}(1-\frac{p}{|\Lambda|})^{\sum_i^{m}|t_{k(i+1)}-t_{k(i)}|} = 2^m(t-s)\sum_{\Delta_1,\dots,\Delta_{m-1}}(1-\frac{p}{|\Lambda|})^{\sum_{i=1}^{m-1}\Delta_i},
$$
nous pouvons échanger la somme et le produit et nous obtenons:
$$\sum_{\Delta_1,\dots,\Delta_{m-1}}(1-\frac{p}{|\Lambda|})^{\sum_{i=1}^{m-1}\Delta_i} = \prod_{i=1}^{m-1}(\sum_{\Delta_i = 1}^{t-s}(1-\frac{p}{|\Lambda|})^{\Delta_i})= \prod_{i=1}^{m-1}\frac{1-(1-\frac{p}{|\Lambda|})^{t-s}}{\frac{p}{|\Lambda|}},
$$
or $(1-x)^\alpha \geqslant 1-\alpha x$ pour tout $x>0,\alpha>0$, nous avons 
$$ \prod_{i=1}^{m-1}\frac{1-(1-\frac{p}{|\Lambda|})^{t-s}}{\frac{p}{|\Lambda|}} \leqslant(t-s)^{m-1}.
$$
Nous avons donc
$$P\left(\begin{array}{c}
x\overset{s,t}{\longleftrightarrow} y \text{ par }\gamma\\
|\gamma| = n
\end{array}\right) \leqslant\sum_{m=0}^{\lfloor\frac{n}{2}\rfloor}\sum_{1=k(0)<\dots<k(m+1)=n} 2^m(3-3p)^{n-m}(t-s)^m\frac{1}{|\Lambda|^m}.
$$
Or le nombre de $1\leqslant k(1)< \dots < k(m) \leqslant n$ est $\binom{n-m+1}{m}$ par lemme 1, nous obtenons:
\begin{align*}P\left(\begin{array}{c}
x\overset{s,t}{\longleftrightarrow} y \text{ par }\gamma\\
|\gamma| = n
\end{array}\right)& \leqslant\sum_{m=0}^{\lfloor\frac{n}{2}\rfloor}\binom{n-m+1}{m} 2^m(3-3p)^{n-m}(t-s)^m\frac{1}{|\Lambda|^m}\\
& \leqslant (3-3p)^{n/2}\sum_{m=0}^{\lfloor\frac{n}{2}\rfloor}\frac{(2n)^m}{m!|\Lambda|^m}(t-s)^m \\
&\leqslant \exp\big(\frac{2n(t-s)}{|\Lambda|}+\frac{n}{2}\ln(3-3p)\big)
\end{align*}
\end{proof}

\end{document}