\documentclass[titlepage,a4paper,12pt]{article}

\usepackage{amsmath,amsfonts,amssymb,amsthm,mathrsfs,enumitem}
\usepackage[french]{babel}
\usepackage{comment}
\usepackage{tikz}
\usepackage[utf8]{inputenc}
\usepackage{centernot}
\usepackage[parfill]{parskip}
\newcounter{d}
\newcounter{t}
\newcounter{p}
\newcounter{c}
\newcounter{a}
\newcounter{l}
\frenchbsetup{StandardLists = true}

\makeatletter
\def\thm@space@setup{%
  \thm@preskip=5pt \thm@postskip=0pt
}
\makeatother

\newtheorem{defi}[d]{Définition}
\newtheorem{prop}[p]{Proposition}
\newtheorem{algo}[a]{Algorithme}
\newtheorem{thm}[t]{Théorème}
\newtheorem{lem}[l]{Lemme}

\begin{document}
\section{chemin espace-temps}
Nous considérons les sommets et les arêtes qui inclus dans une boite $\Lambda(l)$ de taille $2l\times h$. On étudie les chemin espace-tempss fermés dans la boîte formés par la percolation dynamique de paramètre $p$.
\begin{defi}
Une arête-temps est un couple $(e,t)$ où $e$ est une arête de $\mathbb{E}$ et $t$ un nombre réel. 
\end{defi}

Nous définissons une relation d'équivalence de connexion sur l'espace $\mathbb{E}\times \mathbb{R}$ de la manière suivante: nous disons que les arêtes-temps $(e,t)$ et $(f,s)$ sont connectés si $e=f$ ou ($s=t$ et $e,f$ ont une extrémité commune). Nous notons $(e,t)\sim(f,s)$ si l'une des conditions est vérifiée. Un chemin espace-temps est une suite d'arête-temps $(e_i,t_i)_{i\geqslant 0}$ telle que pour tout $i\geqslant 0$, $(e_i,t_i)\sim(e_{i+1},t_{i+1})$. Nous définissons la longueur d'un chemin espace-temps comme le nombre d'arête-temps dans la suite. Nous disons que $(e_i,t_i)$ est un changement de temps si $e_{i+1} = e_i$ et $t_{i+1}\neq t_i$ et nous appelons $[t_i,t_{i+1}]$ si $t_i<t_{i+1}$ ou $[t_{i+1},t_i]$ si $t_i> t_{i+1}$ un intervalle de changement de temps. Désormais, nous considérons les chemins espace-temps avec les changements de temps simples, i.e. si $(e_k,t_k)$ est un changement de temps alors, $e_{k+2} \neq e_k$. 

Nous considérons le processus de percolation dynamique à temps discret et $\omega$ une trajectoire. Un chemin epsace-temps $(e_i,t_i)_{0\leqslant i \leqslant n}$ est dit fermé si pour tout $1\leqslant i\leqslant n$, $e_i$ est fermé à l'instant $t_i$ dans $\omega$. Nous disons que ce chemin espace-temps fermé est d'occurrence disjointe de longueur $n$ avec $m$ changements de temps s'il existe $m$ indices $1\leqslant k(1)< k(2) < \dots < k(m) \leqslant n$ telles que:
\begin{itemize}[label = $\bullet$, leftmargin = *]
\item Les changements de temps arrivent aux instants $t_{k(1)},\dots, t_{k(m)}$, i.e.
$$\forall i\in \{1,\dots, m-1\} \qquad e_{k(i)} = e_{k(i)+1} \quad t_{k(i)+1} =\dots = t_{k(i+1)}.
$$
\item Les arêtes visitées à un instant donné sont 2 à 2 distinctes, i.e.
$$ \forall i,j \in \{1,\dots, n\} \qquad (i\neq j, t_i = t_j) \Rightarrow e_i\neq e_j.
$$
\item Les fermetures d'arêtes arrivent disjointement, i.e, pour tout $i,j\in \{1,\dots, j\}$, $i<j$ tels que $e_i = e_j$, l'une des 3 conditions suivantes est vérifiée:
\begin{itemize}[label=$\bullet$]
\item $j=i+1$ et $i\in \{k(1),\dots, k(m)\}$;
\item $t_i< t_j$ et il existe un instant $s\in ]t_i,t_j[$ tel que $e_j$ est ouverte à $s$ dans $\omega$;
\item $t_j< t_i$ et il existe un instant $s\in ]t_j,t_i[$ tel que $e_j$ est ouverte à $s$ dans $\omega$;
\end{itemize}
\end{itemize}

Soit $x,y$ deux sommets dans $\Lambda(l)$, nous disons qu'un chemin espace-temps $(e_1,t_1),\dots,(e_n,t_n)$ relie $x$ à $y$ si $x$ est l'une extrémité de $e_1$ et $y$ une extrémité de $e_n$.
\begin{prop}
Soit $(e_i,t_i)_{0\leqslant i \leqslant N}$ un chemin espace-temps fermé qui relie $x$ à $y$ dans $\omega$, il existe une fonction $\phi: \{1,\dots, n\}\rightarrow \{1,\dots, N\}$ strictement croissante telle que $(e_{\phi(1)},t_{\phi(1)}),\dots,(e_{\phi(n)},t_{\phi(n)})$ est un chemin espace-temps fermé d'occurrence disjointe qui relie $x$ à $y$ dans $\omega$.
\end{prop}
\begin{proof}
Nous allons montrer cette proposition par récurrence sur la longueur $N$. Supposons que la proposition est vraie pour tout chemin de longueur inférieur à $N$. Considérons maintenant un chemin $$(e_1,t_1),\dots, (e_{N+1},t_{N+1})$$ de longueur $N+1$ qui relie $x$ à $y$. S'il existe un indice $i\leqslant N$ tel que $(e_i,t_i) = (e_{N+1},t_{N+1})$, alors le chemin $$(e_1,t_1),\dots,(e_i,t_i)
$$ est un chemin de longueur $i\leqslant N$ qui relie $x$ à $y$. Par l'hypothèse de récurrence, il existe un chemin extrait d'occurrence disjointe qui relie $x$ à $y$. 
S'il existe un indice $1\leqslant i\leqslant N$ tel que $e_i = e_{N+1}$, et $e_{N+1}$ reste fermée entre $t_i$ et $t_{N+1}$, nous considérons le chemin $$(e_1,t_1), \dots, (e_{i},t_{i}), (e_{N+1},t_{N+1})$$ qui est de longueur inférieur à $N$. Nous appliquons l'hypothèse de récurrence à ce chemin et nous obtenons un chemin extrait d'occurrence disjointe qui relie $x$ à $y$. Si aucun des cas précédents a lieu, nous considérons le chemin $(e_1,t_1),\dots,(e_N,t_N)$ de longueur $N$ et qui relie $x$ à $z$, où $z$ est un voisin de $y$. Par l'hypothèse de récurrence, il existe une fonction strictement croissante $\phi:\{1,\dots,n\}\rightarrow \{1,\dots, N\}$ telle que le chemin extrait $$\gamma(\phi) = (e_{\phi(1)},t_{\phi(1)}),\dots,(e_{\phi(n)},t_{\phi(n)})$$ est un chemin d'occurrence disjointe qui relie $x$ à $z$. 
Si ce chemin n'emprunte pas l'arête $e_{N+1}$, alors nous posons $\phi(n+1) = N+1$ et nous obtenons le chemin extrait souhaité.
Considérons le cas où $\gamma(\phi)$ emprunte l'arête $e_{N+1}$. Supposons tout d'abord que $\gamma(\phi)$ passe par $e_{N+1}$ avant et après $t_{N+1}$. Nous notons $t_-$ (respectivement $t_+$) le dernier (respectivement premier) instant strictement avant (respectivement après) $t_{N+1}$ où $\gamma(\phi)$ visite $e_{N+1}$ et soit $j_-$ (respectivement $j_+$) l'unique indice tel que $t_{\phi(j_-)}= t_-$ et $e_{\phi(j_-)}= e_{N+1}$ (respectivement $t_{\phi(j_+)}= t_+$ et $e_{\phi(j_+)}= e_{N+1}$). Plus formellement, les indices $j_-,j_+$ sont définis par les conditions suivantes:
\begin{align*}
{j_-}< t_{N+1},\quad e_{j_-} = e_{N+1},\quad t_{j_-} = \max \big\{\,t_j:1\leqslant j \leqslant N, e_j = e_{N+1}, t_j < t_{N+1}\,\big\},\\
{j_+}> t_{N+1},\quad e_{j_+} = e_{N+1},\quad t_{j_+} = \min \big\{\,t_j:1\leqslant j \leqslant N, e_j = e_{N+1}, t_j > t_{N+1}\,\big\}.
\end{align*}
Comme le chemin $\gamma(\phi)$ est d'occurrence disjointe et ne contient pas l'arête temps $(e_{N+1},t_{N+1})$, et qu'aucun de ses arêtes-temps $(e_i,t_i)$ de changements de temps n'est restée fermée entre $t_i$ et $t_{N+1}$, nécessairement, l'arête $e_{N+1}$ doit s'ouvrir sur $]t_{j_-},t_{N+1}[$ et sur $]t_{N+1},t_{j_+}[$, nous ajoutons $(e_{N+1},t_{N+1})$ à la fin de $\gamma(\phi)$ et nous obtenons le chemin extrait d'occurrence disjointe qui relie $x$ à $y$.

Maintenant, supposons que $\gamma(\phi)$ visite $e_{N+1}$ uniquement avant $t_{N+1}$. Nous définissons $j_-$ de la même façon que dans le cas précédent. Nécessairement, $e_{N+1}$ s'ouvre entre $t_{j_-}$ et $t_{N+1}$, alors le chemin $$(e_{\phi(1)},t_{\phi(1)}),\dots, (e_{\phi(n)},t_{\phi(n)}),(e_{N+1},t_{N+1})$$ vérifie les conditions voulues.

Enfin, si $\gamma(\phi)$ visite $e_{N+1}$ uniquement après $t_{N+1}$, nous définissons seulement $j_+$ et nous obtenons le chemin extrait voulu de la même manière que le cas précédent.
\end{proof}

\begin{defi}
Un chemin espace-temps $(e_i,t_i)_{0\leqslant i \leqslant n}$ est dit impatient si toute arête de changement de temps $e_k$ est suivi par une arête $e_{k+2}$ qui change son état à l'instant $t_{k+2}$ dans $\omega$, i.e.,
$$ \forall k \in \{1,\dots,n-2\} \quad e_k = e_{k+1} \Rightarrow \omega(e_{k+2},t_{k+2}) \neq \omega(e_{k+2},t_{k+2}^-)
$$
\end{defi}

Nous allons montrer tout chemin espace-temps admet une modification temporelle qui est impatiente. Pour cela, nous introduisons l'algorithme de modification récursive suivante:
\begin{algo} Soit $(e_1,t_1),\dots,(e_n,t_n)$ un chemin espace-temps. Nous allons modifier la première arête $e_1$ du chemin, selon les cas suivants:
\begin{itemize}[label = $\bullet$, leftmargin = *]
\item Si $e_2 \neq e_1$, alors nécessairement $t_1 = t_2$, et nous ne modifions pas $(e_1,t_1)$. Nous recommençons l'algorithme avec le chemin $(e_2,t_2),\dots,(e_n,t_n)$;
\item  Si $t_1< t_2$, soit $\tau_3$ le dernier instant avant $t_2$ où $e_{3}$ se ferme. Si $t_1 \geqslant \tau_3$, nous remplaçons $(e_1,t_1),(e_2,t_2)$ par $(e_1,t_1),(e_3,t_1)$ et nous recommençons avec le chemin $(e_3,t_1),(e_3,t_3),\dots, (e_n,t_n)$. Si $t_1 < \tau_3$, nous remplaçons $(e_1,t_1),(e_2,t_2)$ par $(e_1,t_1),(e_2,\tau_3),(e_3,\tau_3)$. Nous recommençons l'algorithme avec le chemin $(e_3,\tau_3),(e_3,t_3),\dots,(e_n,t_n)$.
\item Si $t_1 > t_2$, soit $\tau_3$ le premier instant après $t_2$ où $e_{3}$ s'ouvre. Si $t_1 \leqslant \tau_3$, nous remplaçons $(e_1,t_1),(e_2,t_2)$ par $(e_1,t_1),(e_3,t_1)$ et nous recommençons avec le chemin $(e_3,t_1),(e_3,t_3),\dots, (e_n,t_n)$. Si $t_1 > \tau_3$, nous remplaçons $(e_1,t_1),(e_2,t_2)$ par $(e_1,t_1),(e_2,\tau_3),(e_3,\tau_3)$. Nous recommençons l'algorithme avec le chemin $(e_3,\tau_3),(e_3,t_3),\dots,(e_n,t_n)$.
\end{itemize}
\end{algo}
Nous remarquons que la longueur du chemin à modifier diminue après chaque itération, donc l'algorithme se termine. Au vu de la définition d'un chemin impatient, nous avons directement la propriété suivante:
\begin{prop}
Soit $(e_1,t_1),\dots,(e_n,t_n)$ un chemin espace-temps qui relie $x$ à $y$, sa modification obtenue selon l'algorithme précédent est impatient qui relie $x$ à $y$ et les intervalles de changement de temps après modification sont inclus dans les intervalles initiaux.
\end{prop}
Nous montrons maintenant qu'un chemin d'occurrence disjointe est toujours d'occurrence disjointe après la modification selon l'algorithme.
\begin{prop} Soit $\gamma = (e_1,t_1),\dots,(e_n,t_n)$ un chemin espace-temps d'occurrence disjointe, nous modifions ce chemin selon algorithme 1. Le chemin obtenu est d'occurrence disjointe et impatient.
\end{prop}
\begin{proof}
Nous vérifions que la condition de l'occurrence disjointe est conservée à chaque étape de l'algorithme. Soit $(e_i,t_i),(e_{i+1},t_{i+1})$ le changement de temps qui est modifié lors d'une itération, et supposons que le chemin visite $e_i$ ou $e_{i+2}$ plus qu'une fois. Supposons aussi que $t_i< t_{i+1}$. Nous examinons les deux résultats possibles de la modification. Si nous obtenons $(e_i,t_i),(e_{i+2},t_i)$, nous devons vérifier qu'il existe un instant entre chaque visite de $e_{i+2}$ et $t_i$ tel que $e_{i+2}$ est ouverte à cette instant. Or $(e_{i+2},t_{i+2})$ est dans $\gamma$ qui est un chemin d'occurrence disjointe, donc $e_{i+2}$ ouvre entre les autres instants de visites et $t_{i+2}$. Vu que l'arête $e_{i+2}$ est fermée entre $t_i$ et $t_{i+2}$, cette propriété est encore vraie pour $t_i$. Si nous avons $(e_i,t_i),(e_{i+1},\tau_{i+2}),(e_{i+2},\tau_{i+2})$ après la modification, nous devons vérifier la condition pour $e_i$ et $e_{i+2}$. Nous rappelons que $e_{i+1}= e_i$ et $\tau_{i+2}$ le dernier instant avant $t_{i+1}$ où $e_{i+2}$ se ferme. Or $e_i$ est fermée entre $t_i$ et $\tau_{i+2}$, donc $e_i$ ouvre entre $\tau_{i+2}$ et les autres instants de visites. De même, $e_{i+2}$ ouvre entre $\tau_{i+2}$ et les autres instants de visites car $e_{i+2}$ est fermée entre $\tau_{i+2}$ et $t_{i+2}$. Enfin, le cas où $t_i> t_{i+1}$ se traite de la même manière.
\end{proof}

\section{Décroissance exponentielle}
Nous démontrons ici que la probabilité d'avoir un chemin espace-temps qui relie deux points décroît exponentiellement vite avec la distance entre les deux points. Nous notons $$\smash{x\overset{s,t}{\longleftrightarrow} y}$$ l'événement: il existe un chemin espace temps fermé qui relie $x$ et $y$ dans l'intervalle de temps $[s,t]$.
Nous commençons par un lemme combinatoire:
\begin{lem} Soit $S(n,m)$ l'ensemble de $m$-uplet d'entiers suivant: $$S(n,m)=\{\,(u_1,\dots,u_m)\in \{1,\dots,n\}^m:\quad \forall 1\leqslant i\leqslant m-1 \quad u_{i+1}>u_i+1 \, \}.$$ Alors $$|S(n,m)| = \binom{n-m+1}{m}.$$
\end{lem}
\begin{proof}
Nous considérons l'application
$$ \Phi : (u_1,\dots,u_m) \rightarrow (u_1,\dots,u_i-i+1,\dots,u_m-m+1).
$$
L'application $\Phi$ est une bijection de $S(n,m)$ sur l'ensemble de $m$-uplet strictement croissant entre $1$ et $n-m+1$, i.e.
$$ \big\{(u_1,\dots,u_m)\in \{1,\dots, n-m+1\}^m:\quad \forall 1\leqslant i \leqslant m-1\quad u_{i+1}> u_i\big\}.
$$ Ce dernier est de cardinal $\binom{n-m+1}{m}$.
\end{proof}

Nous énonçons maintenant notre estimée centrale.
\begin{prop} Soit 
$x,y$ deux points dans $\Lambda$ et $s<t$ deux instants, alors:
$$P\left(\begin{array}{c}
\text{il existe un chemin }\gamma \text{ de longueur }n\\\text{ qui relie }x\text{ à }y \text{ entre }s\text{ et }t
\end{array}\right) \leqslant \exp\left(\frac{2n(t-s)}{|\Lambda|}+\frac{n}{2}\ln(3-3p)\right)$$.
\end{prop}
\begin{proof}
Notons $\mathcal{E}$ l'événement à estimer. Supposons que $\mathcal{E}$ arrive et soit $\gamma$ un chemin espace-temps qui le réalise. Par les propositions précédentes, nous pouvons supposer que $\gamma$ est d'occurrence disjointe et impatient. Notons $(e_1,t_1),\dots,(e_n,t_n)$ les arêtes-temps de $\gamma$, $k(1),\dots,k(m)$ les indices où les changements de temps ont lieu nous notons par convention $k(0) = 1$ et $k(m+1)=n$. Quitte à arrêter $\gamma$ à l'instant où il visite $y$, nous pouvons supposer que $\gamma$ ne se termine pas par un changement de temps, i.e. $k(m)< n-1$. Pour $0\leqslant i \leqslant m$, nous notons $\mathcal{E}_i$ l'événement: il existe un chemin fermé qui relie une extrémité de $e_{k(i)}$ à une extrémité de $e_{k(i+1)}$ à l'instant $t_{k(i+1)}$, $e_{k(i)}$ reste fermée entre $t_{k(i)}$ et $t_{k(i+1)}$ et $e_{k(i)+2}$ se ferme à l'instant $t_{k(i+1)}$. Nous conditionnons $\mathcal{E}$ selon le nombre et les instants de changement de temps puis nous factorisons la probabilité à l'aide de l'inégalité de BK:
\begin{align*}
&P(\mathcal{E})
= \sum_{0\leqslant m \leqslant \frac{n}{2}}\sum_{1\leqslant k(1)<\dots<k(m)\leqslant n}\sum_{t_{k(1)},\dots,t_{k(m)}}P(\mathcal{E}_0\circ\dots\circ \mathcal{E}_m)\\
&\leqslant \sum_{m=0}^{\lfloor\frac{n}{2}\rfloor}\sum_{1=k(0)<\dots<k(m+1)=n}\sum_{t_{k(1)},\dots,t_{k(m)}} \prod_{i=0}^mP(\mathcal{E}_i).
\end{align*}
Nous étudions maintenant chaque terme $P(\mathcal{E}_i)$. Nous notons $x_i,y_i$ les extrémités de l'arête $e_{k(i)}$ dans l'ordre où elles sont traversées par $\gamma$.
$$
P(\mathcal{E}_i) = P\left(\begin{array}{c}
y_i\longleftrightarrow x_{i+1}\text{ à l'instant }t_{k(i+1)}\\
e_{k(i)} \text{ reste fermée entre } t_{k(i)} \text{ et } t_{k(i+1)}\\
e_{k(i)+2} \text{ change d'état à }t_{k(i+1)}
\end{array}\right)
$$
Pour réaliser $y_i\longleftrightarrow x_{i+1}$ à l'instant $t_{k(i+1)}$, il existe un chemin fermé de longueur $k(i+1)-k(i)-1$ car $e_{k(i)} = e_{k(i)+1}$. La probabilité qu'il existe un tel chemin est majorée par $(3-3p)^{k(i+1)-k(i)-1)}$.
Or à chaque instant $t$, nous choisissons une arête uniformément parmi toutes les arêtes de $\Lambda(l)$ et nous déterminons le nouvel état de cette arête selon une loi de Bernoulli  de paramètre $1$, la probabilité que $e_{k(i)}$ reste fermée entre $t_{k(i)}$ et $t_{k(i+1)}$ est donc $(1-\frac{p}{|\Lambda|})^{|t_{k(i+1)}-t_{k(i)}|}$. Enfin, la probabilité que $e_{k(i)+2}$ change son état à l'instant $t_{k(i+1)}$ est $\frac{1}{|\Lambda|}$. Nous obtenons:
$$P(\mathcal{E}_i)\leqslant (3-3p)^{k(i+1)-k(i)-1)}(1-\frac{p}{|\Lambda|})^{|t_{k(i+1)}-t_{k(i)}|}\frac{1}{|\Lambda|}$$
Nous injectons les majorations précédentes dans la première probabilité et nous obtenons:
\begin{multline*}
P(\mathcal{E}_0\circ\dots\circ \mathcal{E}_m) \leqslant \sum_{0\leqslant m\leqslant \frac{n}{2}}\sum_{1\leqslant k(1)<\dots<k(m)<n} \sum_{t_{k(1)},\dots,t_{k(m)}}\\(3-3p)^{n-m}(1-\frac{p}{|\Lambda|})^{\sum_{i=1}^{m}|t_{k(i+1)}-t_{k(i)}|}\frac{1}{|\Lambda|^m}
\end{multline*}

Calculons d'abord la somme sur les instants $t_{k(1)},\dots,t_{k(m)}$, nous posons $\Delta_i = |t_{k(i+1)}-t_{k(i)}|$. Si $m$ et les indices $k(1),\dots,k(m)$ sont fixés, la suite $t_{k(1)},\dots,t_{k(m)}$ est déterminée par la donnée de $t_{k(1)}$, les valeurs de $\Delta_1,\dots,\Delta_{m-1}$ et les signes de $t_{k(i+1)}-t_{k(i)}$, d'où:
$$\sum_{t_{k(1)},\dots,t_{k(m)}}(1-\frac{p}{|\Lambda|})^{\sum_{i=1}^{m}|t_{k(i+1)}-t_{k(i)}|} = 2^{m-1}(t-s)\sum_{1\leqslant\Delta_1,\dots,\Delta_{m-1}\leqslant t-s}(1-\frac{p}{|\Lambda|})^{\Delta_1+\dots+\Delta_{m-1}}.
$$
Nous échangeons la somme et le produit et nous obtenons:
\begin{multline*}\sum_{1\leqslant\Delta_1,\dots,\Delta_{m-1}\leqslant t-s}(1-\frac{p}{|\Lambda|})^{\sum_{i=1}^{m-1}\Delta_i} = \prod_{i=1}^{m-1}\left(\sum_{\Delta_i = 1}^{t-s}\left(1-\frac{p}{|\Lambda|}\right)^{\Delta_i}\right)\\= \prod_{i=1}^{m-1}\left(1-\frac{p}{|\Lambda|}\right)\frac{\displaystyle1-(1-\frac{p}{|\Lambda|})^{t-s}}{\displaystyle\frac{p}{|\Lambda|}}\leqslant \prod_{i=1}^{m-1}\frac{\displaystyle1-(1-\frac{p}{|\Lambda|})^{t-s}}{\displaystyle\frac{p}{|\Lambda|}}.
\end{multline*}
Comme $(1-x)^\alpha \geqslant 1-\alpha x$ pour $0<x<1$ et $\alpha\geqslant 0$, nous avons 
$$ \prod_{i=1}^{m-1}\frac{\displaystyle 1-(1-\frac{p}{|\Lambda|})^{t-s}}{\displaystyle\frac{p}{|\Lambda|}} \leqslant(t-s)^{m-1}.
$$
Nous avons donc
$$P\left(\mathcal{E}\right) \leqslant\sum_{0\leqslant m \leqslant \frac{n}{2}}\sum_{1\leqslant k(1)<\dots<k(m)<n} 2^{m-1}(3-3p)^{n-m}(t-s)^m\frac{1}{|\Lambda|^m}.
$$
Or le nombre de $1\leqslant k(1)< \dots < k(m) \leqslant n$ est $\binom{n-m+1}{m}$ par lemme 1, donc:
\begin{align*}P\left(\mathcal{E}\right)& \leqslant\sum_{0\leqslant m \leqslant\frac{n}{2}}\binom{n-m+1}{m} 2^{m-1}(3-3p)^{n-m}(t-s)^m\frac{1}{|\Lambda|^m}\\
& \leqslant (3-3p)^{n/2}\sum_{m=0}^{\lfloor\frac{n}{2}\rfloor}\frac{(2n)^m}{m!|\Lambda|^m}(t-s)^m \\
&\leqslant \exp\big(\frac{2n(t-s)}{|\Lambda|}+\frac{n}{2}\ln(3-3p)\big).
\end{align*}
\end{proof}

Enfin, utilisons la proposition précédente pour obtenir la décroissance en vitesse exponentielle de la probabilité d'avoir un chemin espace temps qui relie deux points de distance $l$.
\begin{thm} Soit $x,y$ deux points de distance $l$ et $t$ un instant. Alors 
$$P(x\overset{0,t}{\longleftrightarrow} y)\leqslant \exp\big(-C(p)l\big)
$$pour tout $p > \tilde{p}$ où $\tilde{p}$ une constante arbitraire. De plus, $C(p)$ est une constante positive qui tend vers infini quand $p$ tends vers 1.
\end{thm}
\begin{proof}
Remarquons d'abord qu'un chemin espace-temps qui relie $x,y$ est nécessairement de longueur supérieure à $l$. De plus nous pouvons extraire un chemin d'occurrence disjointe et impatient à partir de ce chemin qui relie aussi et $x,y$ entre $0$ et $t$. Or la probabilité qu'un chemin qui vit un temps $t$ est bornée par $\exp(-c(p)t)$, nous pouvons considérer le cas où $t\leqslant \kappa l \leqslant \kappa |\Lambda|$ avec $\kappa$ une constante arbitraire strictement positive.  Nous avons l'inégalité suivante:
\begin{align*}
P(x\overset{0,t}{\longleftrightarrow}& y)=P\left(\begin{array}{c}x\overset{0,t}{\longleftrightarrow} y\text{ par }\gamma\\\gamma \text{ d'occurrence disjointe et impatient}\\ t \leqslant \kappa l\end{array}\right) + \exp(-c(p)\kappa l)\\
\leqslant& \sum_{n\geqslant l}P\left(\begin{array}{c}x\overset{0,t}{\longleftrightarrow} y\text{ par }\gamma\\\gamma \text{ d'occurrence disjointe et impatient}\\ |\gamma| = n, t \leqslant \kappa l\end{array}\right)+ + \exp(-c(p)\kappa l)\\
\leqslant& \sum_{n\geqslant l} \exp\big(2n\kappa+\frac{n}{2}\ln(3-3p)\big) + \exp(-c(p)\kappa l).
\end{align*}
Nous posons $\tilde{p}$ telle que $\kappa+\frac{\ln(3-3\tilde{p})}{2} = 0$. Nous avons donc pour tout $p>\tilde{p}$
\begin{multline*} \sum_{n\geqslant l} \exp\big(2n\kappa+\frac{n}{2}\ln(3-3p)\big) + \exp(-c(p)\kappa l) \\
\leqslant \frac{\exp(l(\kappa+\frac{\ln(3-3p)}{2}))}{1-\exp(\kappa+\frac{\ln(3-3p)}{2})}+\exp(-c(p)\kappa l)\leqslant \exp(-C(p)l)
\end{multline*}
où nous posons $$C(p) = \frac{\min(c(p),-\kappa-\frac{\ln(3-3p)}{2})}{2}$$ qui tend vers infini quand $p$ tend vers 1.
\end{proof}
\end{document}