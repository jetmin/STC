\documentclass[titlepage,a4paper,12pt]{article}

\usepackage{amsmath,amsfonts,amssymb,amsthm,mathrsfs,enumitem}
\usepackage[french]{babel}
\usepackage{comment}
\usepackage{tikz}
\usepackage[utf8]{inputenc}
\usepackage{centernot}
\usepackage[parfill]{parskip}
\newcounter{def}
\newcounter{thm}
\newcounter{prop}
\newcounter{cor}
\frenchbsetup{StandardLists = true}

\makeatletter
\def\thm@space@setup{%
  \thm@preskip=5pt \thm@postskip=0pt
}
\makeatother

\newtheorem{pointTS}[def]{Définition}
\begin{document}
\section{Space-time chemin}

\begin{pointTS}
Une arête-temps est un couple $(e,t)$ où $e$ est une arçete de $\mathbb{E}$ et $t$ un nombre réel. 
\end{pointTS}

Nous définissons une relation d'équivalence de connexion sur l'espace $\mathbb{E}\times \mathbb{R}$ de la manière suivante: nous disons que les arêtes-temps $(e,t)$ et $(f,s)$ sont connectés si $e=f$ ou $(s=t$ et $e\sim f)$. Nous notons $(e,t)\sim(f,s)$ si l'une des conditions est vérifiée. Un space-time chemin est une suite d'arête-temps $(e_i,t_i)_{i\geqslant 0}$ telle que pour tout $i\geqslant 0$, $(e_i,t_i)\sim(e_{i+1},t_{i+1})$. 

Nous appelons un space time chemin d'occurrence disjointe de longueur $n$ avec $m$ changement de temps s'il existe $m$ indices $1\leqslant k(1)< k(2) < \dots < k(m) \leqslant n$ telles que les changements de temps arrivent aux instants $t_{k(1)},\dots, t_{k(m)}$

\end{document}